% Options for packages loaded elsewhere
% Options for packages loaded elsewhere
\PassOptionsToPackage{unicode}{hyperref}
\PassOptionsToPackage{hyphens}{url}
\PassOptionsToPackage{dvipsnames,svgnames,x11names}{xcolor}
%
\documentclass[
  12pt,
  letterpaper,
  12pt,
  letterpaper,
  oneside]{report}
\usepackage{xcolor}
\usepackage[top=1in,bottom=1in,left=1.5in,right=1in]{geometry}
\usepackage{amsmath,amssymb}
\setcounter{secnumdepth}{5}
\usepackage{iftex}
\ifPDFTeX
  \usepackage[T1]{fontenc}
  \usepackage[utf8]{inputenc}
  \usepackage{textcomp} % provide euro and other symbols
\else % if luatex or xetex
  \usepackage{unicode-math} % this also loads fontspec
  \defaultfontfeatures{Scale=MatchLowercase}
  \defaultfontfeatures[\rmfamily]{Ligatures=TeX,Scale=1}
\fi
\usepackage{lmodern}
\ifPDFTeX\else
  % xetex/luatex font selection
\fi
% Use upquote if available, for straight quotes in verbatim environments
\IfFileExists{upquote.sty}{\usepackage{upquote}}{}
\IfFileExists{microtype.sty}{% use microtype if available
  \usepackage[]{microtype}
  \UseMicrotypeSet[protrusion]{basicmath} % disable protrusion for tt fonts
}{}
\usepackage{setspace}
\makeatletter
\@ifundefined{KOMAClassName}{% if non-KOMA class
  \IfFileExists{parskip.sty}{%
    \usepackage{parskip}
  }{% else
    \setlength{\parindent}{0pt}
    \setlength{\parskip}{6pt plus 2pt minus 1pt}}
}{% if KOMA class
  \KOMAoptions{parskip=half}}
\makeatother
% Make \paragraph and \subparagraph free-standing
\makeatletter
\ifx\paragraph\undefined\else
  \let\oldparagraph\paragraph
  \renewcommand{\paragraph}{
    \@ifstar
      \xxxParagraphStar
      \xxxParagraphNoStar
  }
  \newcommand{\xxxParagraphStar}[1]{\oldparagraph*{#1}\mbox{}}
  \newcommand{\xxxParagraphNoStar}[1]{\oldparagraph{#1}\mbox{}}
\fi
\ifx\subparagraph\undefined\else
  \let\oldsubparagraph\subparagraph
  \renewcommand{\subparagraph}{
    \@ifstar
      \xxxSubParagraphStar
      \xxxSubParagraphNoStar
  }
  \newcommand{\xxxSubParagraphStar}[1]{\oldsubparagraph*{#1}\mbox{}}
  \newcommand{\xxxSubParagraphNoStar}[1]{\oldsubparagraph{#1}\mbox{}}
\fi
\makeatother

\usepackage{color}
\usepackage{fancyvrb}
\newcommand{\VerbBar}{|}
\newcommand{\VERB}{\Verb[commandchars=\\\{\}]}
\DefineVerbatimEnvironment{Highlighting}{Verbatim}{commandchars=\\\{\}}
% Add ',fontsize=\small' for more characters per line
\usepackage{framed}
\definecolor{shadecolor}{RGB}{241,243,245}
\newenvironment{Shaded}{\begin{snugshade}}{\end{snugshade}}
\newcommand{\AlertTok}[1]{\textcolor[rgb]{0.68,0.00,0.00}{#1}}
\newcommand{\AnnotationTok}[1]{\textcolor[rgb]{0.37,0.37,0.37}{#1}}
\newcommand{\AttributeTok}[1]{\textcolor[rgb]{0.40,0.45,0.13}{#1}}
\newcommand{\BaseNTok}[1]{\textcolor[rgb]{0.68,0.00,0.00}{#1}}
\newcommand{\BuiltInTok}[1]{\textcolor[rgb]{0.00,0.23,0.31}{#1}}
\newcommand{\CharTok}[1]{\textcolor[rgb]{0.13,0.47,0.30}{#1}}
\newcommand{\CommentTok}[1]{\textcolor[rgb]{0.37,0.37,0.37}{#1}}
\newcommand{\CommentVarTok}[1]{\textcolor[rgb]{0.37,0.37,0.37}{\textit{#1}}}
\newcommand{\ConstantTok}[1]{\textcolor[rgb]{0.56,0.35,0.01}{#1}}
\newcommand{\ControlFlowTok}[1]{\textcolor[rgb]{0.00,0.23,0.31}{\textbf{#1}}}
\newcommand{\DataTypeTok}[1]{\textcolor[rgb]{0.68,0.00,0.00}{#1}}
\newcommand{\DecValTok}[1]{\textcolor[rgb]{0.68,0.00,0.00}{#1}}
\newcommand{\DocumentationTok}[1]{\textcolor[rgb]{0.37,0.37,0.37}{\textit{#1}}}
\newcommand{\ErrorTok}[1]{\textcolor[rgb]{0.68,0.00,0.00}{#1}}
\newcommand{\ExtensionTok}[1]{\textcolor[rgb]{0.00,0.23,0.31}{#1}}
\newcommand{\FloatTok}[1]{\textcolor[rgb]{0.68,0.00,0.00}{#1}}
\newcommand{\FunctionTok}[1]{\textcolor[rgb]{0.28,0.35,0.67}{#1}}
\newcommand{\ImportTok}[1]{\textcolor[rgb]{0.00,0.46,0.62}{#1}}
\newcommand{\InformationTok}[1]{\textcolor[rgb]{0.37,0.37,0.37}{#1}}
\newcommand{\KeywordTok}[1]{\textcolor[rgb]{0.00,0.23,0.31}{\textbf{#1}}}
\newcommand{\NormalTok}[1]{\textcolor[rgb]{0.00,0.23,0.31}{#1}}
\newcommand{\OperatorTok}[1]{\textcolor[rgb]{0.37,0.37,0.37}{#1}}
\newcommand{\OtherTok}[1]{\textcolor[rgb]{0.00,0.23,0.31}{#1}}
\newcommand{\PreprocessorTok}[1]{\textcolor[rgb]{0.68,0.00,0.00}{#1}}
\newcommand{\RegionMarkerTok}[1]{\textcolor[rgb]{0.00,0.23,0.31}{#1}}
\newcommand{\SpecialCharTok}[1]{\textcolor[rgb]{0.37,0.37,0.37}{#1}}
\newcommand{\SpecialStringTok}[1]{\textcolor[rgb]{0.13,0.47,0.30}{#1}}
\newcommand{\StringTok}[1]{\textcolor[rgb]{0.13,0.47,0.30}{#1}}
\newcommand{\VariableTok}[1]{\textcolor[rgb]{0.07,0.07,0.07}{#1}}
\newcommand{\VerbatimStringTok}[1]{\textcolor[rgb]{0.13,0.47,0.30}{#1}}
\newcommand{\WarningTok}[1]{\textcolor[rgb]{0.37,0.37,0.37}{\textit{#1}}}

\usepackage{longtable,booktabs,array}
\usepackage{calc} % for calculating minipage widths
% Correct order of tables after \paragraph or \subparagraph
\usepackage{etoolbox}
\makeatletter
\patchcmd\longtable{\par}{\if@noskipsec\mbox{}\fi\par}{}{}
\makeatother
% Allow footnotes in longtable head/foot
\IfFileExists{footnotehyper.sty}{\usepackage{footnotehyper}}{\usepackage{footnote}}
\makesavenoteenv{longtable}
\usepackage{graphicx}
\makeatletter
\newsavebox\pandoc@box
\newcommand*\pandocbounded[1]{% scales image to fit in text height/width
  \sbox\pandoc@box{#1}%
  \Gscale@div\@tempa{\textheight}{\dimexpr\ht\pandoc@box+\dp\pandoc@box\relax}%
  \Gscale@div\@tempb{\linewidth}{\wd\pandoc@box}%
  \ifdim\@tempb\p@<\@tempa\p@\let\@tempa\@tempb\fi% select the smaller of both
  \ifdim\@tempa\p@<\p@\scalebox{\@tempa}{\usebox\pandoc@box}%
  \else\usebox{\pandoc@box}%
  \fi%
}
% Set default figure placement to htbp
\def\fps@figure{htbp}
\makeatother


% definitions for citeproc citations
\NewDocumentCommand\citeproctext{}{}
\NewDocumentCommand\citeproc{mm}{%
  \begingroup\def\citeproctext{#2}\cite{#1}\endgroup}
\makeatletter
 % allow citations to break across lines
 \let\@cite@ofmt\@firstofone
 % avoid brackets around text for \cite:
 \def\@biblabel#1{}
 \def\@cite#1#2{{#1\if@tempswa , #2\fi}}
\makeatother
\newlength{\cslhangindent}
\setlength{\cslhangindent}{1.5em}
\newlength{\csllabelwidth}
\setlength{\csllabelwidth}{3em}
\newenvironment{CSLReferences}[2] % #1 hanging-indent, #2 entry-spacing
 {\begin{list}{}{%
  \setlength{\itemindent}{0pt}
  \setlength{\leftmargin}{0pt}
  \setlength{\parsep}{0pt}
  % turn on hanging indent if param 1 is 1
  \ifodd #1
   \setlength{\leftmargin}{\cslhangindent}
   \setlength{\itemindent}{-1\cslhangindent}
  \fi
  % set entry spacing
  \setlength{\itemsep}{#2\baselineskip}}}
 {\end{list}}
\usepackage{calc}
\newcommand{\CSLBlock}[1]{\hfill\break\parbox[t]{\linewidth}{\strut\ignorespaces#1\strut}}
\newcommand{\CSLLeftMargin}[1]{\parbox[t]{\csllabelwidth}{\strut#1\strut}}
\newcommand{\CSLRightInline}[1]{\parbox[t]{\linewidth - \csllabelwidth}{\strut#1\strut}}
\newcommand{\CSLIndent}[1]{\hspace{\cslhangindent}#1}



\setlength{\emergencystretch}{3em} % prevent overfull lines

\providecommand{\tightlist}{%
  \setlength{\itemsep}{0pt}\setlength{\parskip}{0pt}}



 


% Simple preamble for dissertation PDF
% Compatible with standard LaTeX report class

% Essential packages
\usepackage{amsmath}
\usepackage{amsfonts}
\usepackage{amssymb}
\usepackage{graphicx}
\usepackage{longtable}
\usepackage{booktabs}
\usepackage{array}
\usepackage{multirow}
\usepackage{setspace}

% Set double spacing
\doublespacing

% Custom title page formatting
\usepackage{titling}
\pretitle{\begin{center}\Large\bfseries}
\posttitle{\end{center}}
\preauthor{\begin{center}\large}
\postauthor{\end{center}}
\predate{\begin{center}\large}
\postdate{\end{center}}

% Custom commands
\newcommand{\todo}[1]{\textcolor{red}{TODO: #1}}

% Chapter/section formatting
\usepackage{titlesec}
\titleformat{\chapter}[display]
  {\normalfont\huge\bfseries}{\chaptertitlename\ \thechapter}{20pt}{\Huge}
\titlespacing*{\chapter}{0pt}{50pt}{40pt}
\makeatletter
\@ifpackageloaded{bookmark}{}{\usepackage{bookmark}}
\makeatother
\makeatletter
\@ifpackageloaded{caption}{}{\usepackage{caption}}
\AtBeginDocument{%
\ifdefined\contentsname
  \renewcommand*\contentsname{Table of contents}
\else
  \newcommand\contentsname{Table of contents}
\fi
\ifdefined\listfigurename
  \renewcommand*\listfigurename{List of Figures}
\else
  \newcommand\listfigurename{List of Figures}
\fi
\ifdefined\listtablename
  \renewcommand*\listtablename{List of Tables}
\else
  \newcommand\listtablename{List of Tables}
\fi
\ifdefined\figurename
  \renewcommand*\figurename{Figure}
\else
  \newcommand\figurename{Figure}
\fi
\ifdefined\tablename
  \renewcommand*\tablename{Table}
\else
  \newcommand\tablename{Table}
\fi
}
\@ifpackageloaded{float}{}{\usepackage{float}}
\floatstyle{ruled}
\@ifundefined{c@chapter}{\newfloat{codelisting}{h}{lop}}{\newfloat{codelisting}{h}{lop}[chapter]}
\floatname{codelisting}{Listing}
\newcommand*\listoflistings{\listof{codelisting}{List of Listings}}
\makeatother
\makeatletter
\makeatother
\makeatletter
\@ifpackageloaded{caption}{}{\usepackage{caption}}
\@ifpackageloaded{subcaption}{}{\usepackage{subcaption}}
\makeatother
\usepackage{bookmark}
\IfFileExists{xurl.sty}{\usepackage{xurl}}{} % add URL line breaks if available
\urlstyle{same}
\hypersetup{
  pdftitle={Dissertation Title},
  pdfauthor={Author Name},
  colorlinks=true,
  linkcolor={blue},
  filecolor={Maroon},
  citecolor={Blue},
  urlcolor={Blue},
  pdfcreator={LaTeX via pandoc}}


\title{Dissertation Title}
\author{Author Name}
\date{2025-07-09}
\begin{document}
\maketitle

\renewcommand*\contentsname{Table of contents}
{
\hypersetup{linkcolor=}
\setcounter{tocdepth}{2}
\tableofcontents
}
\listoffigures
\listoftables

\setstretch{2}
\bookmarksetup{startatroot}

\chapter*{Dissertation}\label{dissertation}
\addcontentsline{toc}{chapter}{Dissertation}

\markboth{Dissertation}{Dissertation}

This is the main dissertation document. It will be compiled to both HTML
(for web viewing) and PDF (using the University of Colorado thesis
class).

\section*{Abstract}\label{abstract}
\addcontentsline{toc}{section}{Abstract}

\markright{Abstract}

{[}Abstract content goes here{]}

\section*{Acknowledgments}\label{acknowledgments}
\addcontentsline{toc}{section}{Acknowledgments}

\markright{Acknowledgments}

{[}Acknowledgments content goes here{]}

\section*{Table of Contents}\label{table-of-contents}
\addcontentsline{toc}{section}{Table of Contents}

\markright{Table of Contents}

The table of contents will be automatically generated for both HTML and
PDF formats.

\bookmarksetup{startatroot}

\chapter{Introduction}\label{introduction}

This is the introduction chapter of the dissertation.

\section{Background}\label{background}

{[}Background content{]}

\section{Research Questions}\label{research-questions}

{[}Research questions content{]}

\section{Significance of the Study}\label{significance-of-the-study}

{[}Significance content{]}

\bookmarksetup{startatroot}

\chapter{Literature Review}\label{literature-review}

This chapter provides a comprehensive review of the existing literature
related to the research topic.

\section{Overview}\label{overview}

The literature review examines current research, identifies gaps in
knowledge, and establishes the theoretical foundation for this study.

\section{Theoretical Framework}\label{theoretical-framework}

\subsection{Key Theories}\label{key-theories}

Discuss the main theoretical perspectives that inform this research:

\begin{itemize}
\tightlist
\item
  \textbf{Theory 1}: Brief description and relevance
\item
  \textbf{Theory 2}: Brief description and relevance
\item
  \textbf{Theory 3}: Brief description and relevance
\end{itemize}

\section{Review of Empirical Studies}\label{review-of-empirical-studies}

\subsection{Early Research (Year
Range)}\label{early-research-year-range}

Review foundational studies in the field:

\begin{itemize}
\tightlist
\item
  \textbf{Author (Year)}: Brief summary of findings and methodology
\item
  \textbf{Author (Year)}: Brief summary of findings and methodology
\end{itemize}

\subsection{Recent Developments (Year
Range)}\label{recent-developments-year-range}

Examine more recent contributions:

\begin{itemize}
\tightlist
\item
  \textbf{Author (Year)}: Brief summary of findings and methodology
  (Author, 2024)
\item
  \textbf{Author (Year)}: Brief summary of findings and methodology
  (Sample, 2023)
\end{itemize}

\section{Methodological
Considerations}\label{methodological-considerations}

\subsection{Research Approaches}\label{research-approaches}

Discussion of various methodological approaches used in the literature:

\begin{itemize}
\tightlist
\item
  \textbf{Quantitative approaches}: Common methods and their
  applications
\item
  \textbf{Qualitative approaches}: Common methods and their applications
\item
  \textbf{Mixed methods}: Integration of approaches
\end{itemize}

\subsection{Limitations in Existing
Research}\label{limitations-in-existing-research}

Identify common limitations across studies:

\begin{itemize}
\tightlist
\item
  Sample size constraints
\item
  Methodological limitations
\item
  Temporal limitations
\item
  Generalizability concerns
\end{itemize}

\section{Gaps in the Literature}\label{gaps-in-the-literature}

\subsection{Identified Gaps}\label{identified-gaps}

\begin{enumerate}
\def\labelenumi{\arabic{enumi}.}
\tightlist
\item
  \textbf{Gap 1}: Description of what is missing in current research
\item
  \textbf{Gap 2}: Description of methodological or theoretical gaps
\item
  \textbf{Gap 3}: Description of empirical gaps
\end{enumerate}

\subsection{Justification for Current
Study}\label{justification-for-current-study}

Explain how this dissertation addresses the identified gaps:

\begin{itemize}
\tightlist
\item
  How this research fills theoretical gaps
\item
  Methodological innovations proposed
\item
  Empirical contributions expected
\end{itemize}

\section{Conceptual Model}\label{conceptual-model}

\subsection{Framework Development}\label{framework-development}

Based on the literature review, present the conceptual framework that
guides this study:

\begin{itemize}
\tightlist
\item
  Key variables and constructs
\item
  Hypothesized relationships
\item
  Theoretical justification
\end{itemize}

\section{Chapter Summary}\label{chapter-summary}

Summarize the key findings from the literature review:

\begin{itemize}
\tightlist
\item
  Main themes identified
\item
  Theoretical contributions
\item
  Empirical findings
\item
  Implications for current research
\end{itemize}

The literature review establishes that {[}brief summary of what
literature shows{]} and identifies {[}key gaps{]} that this dissertation
will address through {[}brief description of approach{]}.

\bookmarksetup{startatroot}

\chapter{Methodology}\label{methodology}

This chapter outlines the research methodology employed in this study,
including the research design, data collection procedures, and
analytical approaches.

\section{Research Design}\label{research-design}

\subsection{Research Paradigm}\label{research-paradigm}

This study is grounded in a {[}paradigm{]} approach, which emphasizes
{[}key characteristics of the paradigm{]}.

\subsection{Research Questions}\label{research-questions-1}

The following research questions guide this investigation:

\begin{enumerate}
\def\labelenumi{\arabic{enumi}.}
\tightlist
\item
  \textbf{RQ1}: Primary research question
\item
  \textbf{RQ2}: Secondary research question\\
\item
  \textbf{RQ3}: Additional research question
\end{enumerate}

\subsection{Hypotheses}\label{hypotheses}

Based on the literature review and theoretical framework, the following
hypotheses are proposed:

\begin{itemize}
\tightlist
\item
  \textbf{H1}: Hypothesis statement with expected direction
\item
  \textbf{H2}: Hypothesis statement with expected direction
\item
  \textbf{H3}: Hypothesis statement with expected direction
\end{itemize}

\section{Study Design}\label{study-design}

\subsection{Overall Approach}\label{overall-approach}

This study employs a {[}quantitative/qualitative/mixed methods{]} design
to {[}brief description of what the design accomplishes{]}.

\subsection{Study Type}\label{study-type}

\begin{itemize}
\tightlist
\item
  \textbf{Design}:
  {[}Cross-sectional/Longitudinal/Experimental/Quasi-experimental{]}
\item
  \textbf{Timeline}: {[}Duration of study{]}
\item
  \textbf{Setting}: {[}Where the study takes place{]}
\end{itemize}

\section{Participants}\label{participants}

\subsection{Population}\label{population}

The target population for this study consists of {[}description of
population{]}.

\subsection{Sampling Strategy}\label{sampling-strategy}

\subsubsection{Sampling Method}\label{sampling-method}

\begin{itemize}
\tightlist
\item
  \textbf{Approach}: {[}Probability/Non-probability sampling{]}
\item
  \textbf{Specific method}:
  {[}Random/Stratified/Convenience/Purposive{]}
\item
  \textbf{Rationale}: Justification for sampling approach
\end{itemize}

\subsubsection{Sample Size}\label{sample-size}

\begin{itemize}
\tightlist
\item
  \textbf{Target sample size}: N = {[}number{]}
\item
  \textbf{Power analysis}: Based on {[}effect size{]}, α = {[}alpha
  level{]}, power = {[}power level{]}
\item
  \textbf{Justification}: {[}Rationale for sample size{]}
\end{itemize}

\subsection{Inclusion/Exclusion
Criteria}\label{inclusionexclusion-criteria}

\subsubsection{Inclusion Criteria}\label{inclusion-criteria}

\begin{itemize}
\tightlist
\item
  Criterion 1
\item
  Criterion 2
\item
  Criterion 3
\end{itemize}

\subsubsection{Exclusion Criteria}\label{exclusion-criteria}

\begin{itemize}
\tightlist
\item
  Criterion 1
\item
  Criterion 2
\item
  Criterion 3
\end{itemize}

\section{Data Collection}\label{data-collection}

\subsection{Instrumentation}\label{instrumentation}

\subsubsection{Primary Instrument}\label{primary-instrument}

\textbf{{[}Instrument Name{]}} - \textbf{Purpose}: What the instrument
measures - \textbf{Format}: {[}Likert scale/Open-ended/Multiple
choice{]} - \textbf{Items}: Number of items and sample items -
\textbf{Reliability}: Cronbach's α = {[}value{]} (if applicable) -
\textbf{Validity}: Evidence of validity

\subsubsection{Secondary Instruments}\label{secondary-instruments}

\textbf{{[}Additional Instrument Name{]}} - \textbf{Purpose}: What the
instrument measures - \textbf{Psychometric properties}: Reliability and
validity information

\subsection{Data Collection
Procedures}\label{data-collection-procedures}

\subsubsection{Phase 1: Preparation}\label{phase-1-preparation}

\begin{enumerate}
\def\labelenumi{\arabic{enumi}.}
\tightlist
\item
  IRB approval process
\item
  Pilot testing of instruments
\item
  Recruitment strategy implementation
\end{enumerate}

\subsubsection{Phase 2: Data Collection}\label{phase-2-data-collection}

\begin{enumerate}
\def\labelenumi{\arabic{enumi}.}
\tightlist
\item
  Participant recruitment
\item
  Informed consent procedures
\item
  Data collection protocol
\item
  Quality assurance measures
\end{enumerate}

\subsubsection{Phase 3: Data Management}\label{phase-3-data-management}

\begin{enumerate}
\def\labelenumi{\arabic{enumi}.}
\tightlist
\item
  Data entry procedures
\item
  Data cleaning protocols
\item
  Data security measures
\end{enumerate}

\section{Variables}\label{variables}

\subsection{Independent Variables}\label{independent-variables}

\begin{itemize}
\tightlist
\item
  \textbf{Variable 1}: {[}Operational definition and measurement{]}
\item
  \textbf{Variable 2}: {[}Operational definition and measurement{]}
\end{itemize}

\subsection{Dependent Variables}\label{dependent-variables}

\begin{itemize}
\tightlist
\item
  \textbf{Variable 1}: {[}Operational definition and measurement{]}
\item
  \textbf{Variable 2}: {[}Operational definition and measurement{]}
\end{itemize}

\subsection{Control Variables}\label{control-variables}

\begin{itemize}
\tightlist
\item
  \textbf{Variable 1}: {[}Operational definition and measurement{]}
\item
  \textbf{Variable 2}: {[}Operational definition and measurement{]}
\end{itemize}

\section{Data Analysis}\label{data-analysis}

\subsection{Analytical Framework}\label{analytical-framework}

The data analysis follows a {[}sequential/concurrent/embedded{]}
approach using {[}software/statistical package{]}.

\subsection{Descriptive Analysis}\label{descriptive-analysis}

\begin{itemize}
\tightlist
\item
  \textbf{Participant characteristics}: Demographic analysis
\item
  \textbf{Variable distributions}: Measures of central tendency and
  variability
\item
  \textbf{Missing data}: Assessment and handling procedures
\end{itemize}

\subsection{Inferential Analysis}\label{inferential-analysis}

\subsubsection{Quantitative Analysis}\label{quantitative-analysis}

\textbf{Primary Analysis} - \textbf{Statistical test}:
{[}t-test/ANOVA/regression/SEM{]} - \textbf{Assumptions}: Testing and
verification - \textbf{Effect size}: Calculation and interpretation

\textbf{Secondary Analysis} - \textbf{Additional tests}:
{[}Correlation/mediation/moderation{]} - \textbf{Multiple comparisons}:
Correction procedures if applicable

\subsubsection{Qualitative Analysis (if
applicable)}\label{qualitative-analysis-if-applicable}

\textbf{Coding Procedures} - \textbf{Initial coding}: Open coding
approach - \textbf{Axial coding}: Category development -
\textbf{Selective coding}: Theme identification

\textbf{Reliability and Validity} - \textbf{Inter-rater reliability}:
Procedures and statistics - \textbf{Member checking}: Validation
approach - \textbf{Triangulation}: Multiple data sources

\subsection{Software and Tools}\label{software-and-tools}

\begin{itemize}
\tightlist
\item
  \textbf{Statistical software}: {[}R/SPSS/SAS/Stata{]}
\item
  \textbf{Qualitative software}: {[}NVivo/Atlas.ti/MaxQDA{]} (if
  applicable)
\item
  \textbf{Additional tools}: {[}Mplus/LISREL{]} (if applicable)
\end{itemize}

\section{Ethical Considerations}\label{ethical-considerations}

\subsection{IRB Approval}\label{irb-approval}

\begin{itemize}
\tightlist
\item
  \textbf{Institution}: {[}University name{]}
\item
  \textbf{Approval number}: {[}IRB number{]}
\item
  \textbf{Date}: {[}Approval date{]}
\end{itemize}

\subsection{Informed Consent}\label{informed-consent}

\begin{itemize}
\tightlist
\item
  \textbf{Consent process}: Description of informed consent procedures
\item
  \textbf{Participant rights}: Information about withdrawal and
  confidentiality
\item
  \textbf{Risk mitigation}: Procedures to minimize potential risks
\end{itemize}

\subsection{Data Protection}\label{data-protection}

\begin{itemize}
\tightlist
\item
  \textbf{Confidentiality}: Measures to protect participant identity
\item
  \textbf{Data storage}: Secure storage and access procedures
\item
  \textbf{Data retention}: Timeline for data retention and disposal
\end{itemize}

\section{Limitations}\label{limitations}

\subsection{Methodological
Limitations}\label{methodological-limitations}

\begin{enumerate}
\def\labelenumi{\arabic{enumi}.}
\tightlist
\item
  \textbf{Design limitations}: Inherent constraints of the chosen design
\item
  \textbf{Sampling limitations}: Potential issues with generalizability
\item
  \textbf{Measurement limitations}: Instrument-related constraints
\end{enumerate}

\subsection{Practical Limitations}\label{practical-limitations}

\begin{enumerate}
\def\labelenumi{\arabic{enumi}.}
\tightlist
\item
  \textbf{Resource constraints}: Time, funding, or access limitations
\item
  \textbf{External factors}: Uncontrollable variables that may affect
  results
\item
  \textbf{Response limitations}: Potential for social desirability or
  response bias
\end{enumerate}

\section{Chapter Summary}\label{chapter-summary-1}

This methodology chapter outlined the
\hyperref[research-design]{research design} approach used to address the
research questions. The study employs {[}key methodological features{]}
with {[}sample description{]} to investigate {[}brief research focus{]}.
The analytical approach combines {[}analytical methods{]} to provide
comprehensive answers to the research questions while addressing
potential limitations through {[}mitigation strategies{]}.

\bookmarksetup{startatroot}

\chapter{Results}\label{results}

This chapter presents the findings from the data analysis, organized by
research question and hypothesis. The results are presented objectively
without interpretation, which will be addressed in the Discussion
chapter.

\section{Data Collection Summary}\label{data-collection-summary}

\subsection{Response Rate}\label{response-rate}

\begin{itemize}
\tightlist
\item
  \textbf{Total participants contacted}: N = {[}number{]}
\item
  \textbf{Completed responses}: N = {[}number{]}
\item
  \textbf{Response rate}: {[}percentage{]}\%
\item
  \textbf{Excluded responses}: N = {[}number{]} (reasons: {[}list
  reasons{]})
\item
  \textbf{Final sample size}: N = {[}number{]}
\end{itemize}

\subsection{Data Quality}\label{data-quality}

\begin{itemize}
\tightlist
\item
  \textbf{Missing data}: {[}percentage{]}\% overall
\item
  \textbf{Complete cases}: N = {[}number{]}
\item
  \textbf{Data cleaning procedures}: {[}brief summary of procedures
  applied{]}
\end{itemize}

\section{Descriptive Statistics}\label{descriptive-statistics}

\subsection{Participant
Characteristics}\label{participant-characteristics}

\subsubsection{Demographic Information}\label{demographic-information}

\begin{table}

\caption{\label{tbl-demographics}Participant Demographics}

\centering{

\begin{Shaded}
\begin{Highlighting}[]
\CommentTok{\# Demographic table will be generated here}
\end{Highlighting}
\end{Shaded}

}

\end{table}%

\textbf{Table 4.1} presents the demographic characteristics of the
sample:

\begin{itemize}
\tightlist
\item
  \textbf{Age}: M = {[}mean{]}, SD = {[}standard deviation{]}, Range =
  {[}range{]}
\item
  \textbf{Gender}: {[}breakdown by categories{]}
\item
  \textbf{Education level}: {[}breakdown by categories{]}
\item
  \textbf{Other relevant demographics}: {[}additional characteristics{]}
\end{itemize}

\subsection{Variable Distributions}\label{variable-distributions}

\subsubsection{Primary Variables}\label{primary-variables}

\begin{table}

\caption{\label{tbl-descriptives}Descriptive Statistics for Primary
Variables}

\centering{

\begin{Shaded}
\begin{Highlighting}[]
\CommentTok{\# Descriptive statistics table will be generated here}
\end{Highlighting}
\end{Shaded}

}

\end{table}%

\textbf{Table 4.2} shows descriptive statistics for all primary
variables:

\begin{itemize}
\tightlist
\item
  \textbf{Variable 1}: M = {[}mean{]}, SD = {[}standard deviation{]},
  Skewness = {[}value{]}, Kurtosis = {[}value{]}
\item
  \textbf{Variable 2}: M = {[}mean{]}, SD = {[}standard deviation{]},
  Skewness = {[}value{]}, Kurtosis = {[}value{]}
\item
  \textbf{Variable 3}: M = {[}mean{]}, SD = {[}standard deviation{]},
  Skewness = {[}value{]}, Kurtosis = {[}value{]}
\end{itemize}

\subsubsection{Distribution Assessment}\label{distribution-assessment}

\begin{Shaded}
\begin{Highlighting}[]
\CommentTok{\# Histogram/density plots will be generated here}
\end{Highlighting}
\end{Shaded}

\textbf{Figure 4.1} displays the distributions of primary variables.
{[}Brief description of distribution patterns and any notable
features{]}.

\section{Assumption Testing}\label{assumption-testing}

\subsection{Statistical Assumptions}\label{statistical-assumptions}

\subsubsection{Normality}\label{normality}

\begin{itemize}
\tightlist
\item
  \textbf{Shapiro-Wilk tests}: {[}Results summary{]}
\item
  \textbf{Q-Q plots}: {[}Visual assessment results{]}
\item
  \textbf{Kolmogorov-Smirnov tests}: {[}Results if applicable{]}
\end{itemize}

\subsubsection{Linearity}\label{linearity}

\begin{Shaded}
\begin{Highlighting}[]
\CommentTok{\# Scatterplot matrix will be generated here}
\end{Highlighting}
\end{Shaded}

\textbf{Figure 4.2} shows scatterplots examining linearity assumptions.
{[}Brief description of linearity assessment{]}.

\subsubsection{Homoscedasticity}\label{homoscedasticity}

\begin{itemize}
\tightlist
\item
  \textbf{Levene's test}: F({[}df1, df2{]}) = {[}value{]}, p =
  {[}p-value{]}
\item
  \textbf{Breusch-Pagan test}: χ²({[}df{]}) = {[}value{]}, p =
  {[}p-value{]}
\end{itemize}

\subsubsection{Multicollinearity}\label{multicollinearity}

\begin{table}

\caption{\label{tbl-correlation}Correlation Matrix and Multicollinearity
Statistics}

\centering{

\begin{Shaded}
\begin{Highlighting}[]
\CommentTok{\# Correlation matrix with VIF values will be generated here}
\end{Highlighting}
\end{Shaded}

}

\end{table}%

\textbf{Table 4.3} presents the correlation matrix and variance
inflation factors (VIF). All VIF values were below {[}threshold{]},
indicating no multicollinearity concerns.

\section{Research Question 1: {[}Question
Text{]}}\label{research-question-1-question-text}

\subsection{Hypothesis 1: {[}Hypothesis
Statement{]}}\label{hypothesis-1-hypothesis-statement}

\subsubsection{Primary Analysis}\label{primary-analysis}

\begin{table}

\caption{\label{tbl-rq1-primary}Primary Analysis Results for Research
Question 1}

\centering{

\begin{Shaded}
\begin{Highlighting}[]
\CommentTok{\# Analysis results table will be generated here}
\end{Highlighting}
\end{Shaded}

}

\end{table}%

\textbf{Primary Statistical Test}: {[}Test name{]} - \textbf{Test
statistic}: {[}statistic{]} = {[}value{]} - \textbf{Degrees of freedom}:
{[}df{]} - \textbf{p-value}: p = {[}value{]} - \textbf{Effect size}:
{[}effect size measure{]} = {[}value{]} - \textbf{Confidence interval}:
{[}CI range{]}

\textbf{Result}: {[}Brief statement of statistical significance and
direction{]}

\subsubsection{Secondary Analysis}\label{secondary-analysis}

\begin{Shaded}
\begin{Highlighting}[]
\CommentTok{\# Relevant plot/graph will be generated here}
\end{Highlighting}
\end{Shaded}

\textbf{Figure 4.3} illustrates {[}description of what the figure
shows{]}. {[}Brief description of visual patterns{]}.

\textbf{Post-hoc Analysis} (if applicable): - \textbf{Multiple
comparisons}: {[}Results of post-hoc tests{]} - \textbf{Pairwise
comparisons}: {[}Specific comparisons and results{]}

\section{Research Question 2: {[}Question
Text{]}}\label{research-question-2-question-text}

\subsection{Hypothesis 2: {[}Hypothesis
Statement{]}}\label{hypothesis-2-hypothesis-statement}

\subsubsection{Primary Analysis}\label{primary-analysis-1}

\begin{table}

\caption{\label{tbl-rq2-primary}Primary Analysis Results for Research
Question 2}

\centering{

\begin{Shaded}
\begin{Highlighting}[]
\CommentTok{\# Analysis results table will be generated here}
\end{Highlighting}
\end{Shaded}

}

\end{table}%

\textbf{Primary Statistical Test}: {[}Test name{]} - \textbf{Test
statistic}: {[}statistic{]} = {[}value{]} - \textbf{Degrees of freedom}:
{[}df{]} - \textbf{p-value}: p = {[}value{]} - \textbf{Effect size}:
{[}effect size measure{]} = {[}value{]} - \textbf{Confidence interval}:
{[}CI range{]}

\textbf{Result}: {[}Brief statement of statistical significance and
direction{]}

\subsubsection{Model Fit (if applicable)}\label{model-fit-if-applicable}

\begin{table}

\caption{\label{tbl-model-fit}Model Fit Statistics}

\centering{

\begin{Shaded}
\begin{Highlighting}[]
\CommentTok{\# Model fit indices will be generated here}
\end{Highlighting}
\end{Shaded}

}

\end{table}%

\textbf{Table 4.5} presents model fit statistics: - \textbf{Chi-square}:
χ²({[}df{]}) = {[}value{]}, p = {[}p-value{]} - \textbf{CFI}:
{[}value{]} - \textbf{TLI}: {[}value{]} - \textbf{RMSEA}: {[}value{]}
(90\% CI: {[}lower, upper{]}) - \textbf{SRMR}: {[}value{]}

\section{Research Question 3: {[}Question
Text{]}}\label{research-question-3-question-text}

\subsection{Hypothesis 3: {[}Hypothesis
Statement{]}}\label{hypothesis-3-hypothesis-statement}

\subsubsection{Primary Analysis}\label{primary-analysis-2}

\begin{table}

\caption{\label{tbl-rq3-primary}Primary Analysis Results for Research
Question 3}

\centering{

\begin{Shaded}
\begin{Highlighting}[]
\CommentTok{\# Analysis results table will be generated here}
\end{Highlighting}
\end{Shaded}

}

\end{table}%

\textbf{Primary Statistical Test}: {[}Test name{]} - \textbf{Test
statistic}: {[}statistic{]} = {[}value{]} - \textbf{Degrees of freedom}:
{[}df{]} - \textbf{p-value}: p = {[}value{]} - \textbf{Effect size}:
{[}effect size measure{]} = {[}value{]} - \textbf{Confidence interval}:
{[}CI range{]}

\textbf{Result}: {[}Brief statement of statistical significance and
direction{]}

\subsubsection{Mediation/Moderation Analysis (if
applicable)}\label{mediationmoderation-analysis-if-applicable}

\begin{Shaded}
\begin{Highlighting}[]
\CommentTok{\# Path diagram will be generated here}
\end{Highlighting}
\end{Shaded}

\textbf{Figure 4.4} shows the path model for the mediation analysis.

\textbf{Mediation Results}: - \textbf{Direct effect}: b = {[}value{]},
SE = {[}value{]}, p = {[}p-value{]} - \textbf{Indirect effect}: b =
{[}value{]}, SE = {[}value{]}, 95\% CI = {[}lower, upper{]} -
\textbf{Total effect}: b = {[}value{]}, SE = {[}value{]}, p =
{[}p-value{]}

\section{Additional Analyses}\label{additional-analyses}

\subsection{Exploratory Analysis}\label{exploratory-analysis}

\subsubsection{Unexpected Findings}\label{unexpected-findings}

{[}Description of any unexpected or interesting findings that emerged
during analysis{]}

\begin{Shaded}
\begin{Highlighting}[]
\CommentTok{\# Additional visualizations will be generated here}
\end{Highlighting}
\end{Shaded}

\subsubsection{Sensitivity Analysis}\label{sensitivity-analysis}

\begin{itemize}
\tightlist
\item
  \textbf{Outlier analysis}: {[}Results of outlier detection and
  impact{]}
\item
  \textbf{Robustness checks}: {[}Results of alternative analytical
  approaches{]}
\item
  \textbf{Missing data sensitivity}: {[}Results of different missing
  data handling approaches{]}
\end{itemize}

\subsection{Subgroup Analysis}\label{subgroup-analysis}

\subsubsection{Analysis by {[}Grouping
Variable{]}}\label{analysis-by-grouping-variable}

\begin{table}

\caption{\label{tbl-subgroup}Subgroup Analysis Results}

\centering{

\begin{Shaded}
\begin{Highlighting}[]
\CommentTok{\# Subgroup analysis table will be generated here}
\end{Highlighting}
\end{Shaded}

}

\end{table}%

\textbf{Table 4.7} presents results stratified by {[}grouping
variable{]}: - \textbf{Group 1}: {[}Results summary{]} - \textbf{Group
2}: {[}Results summary{]} - \textbf{Group comparison}: {[}Statistical
test results{]}

\section{Qualitative Results (if
applicable)}\label{qualitative-results-if-applicable}

\subsection{Thematic Analysis}\label{thematic-analysis}

\subsubsection{Theme 1: {[}Theme Name{]}}\label{theme-1-theme-name}

{[}Description of theme with supporting quotes{]}

\begin{quote}
``Representative quote from participant'' (Participant ID)
\end{quote}

\textbf{Sub-themes}: - \textbf{Sub-theme 1.1}: {[}Description{]} -
\textbf{Sub-theme 1.2}: {[}Description{]}

\subsubsection{Theme 2: {[}Theme Name{]}}\label{theme-2-theme-name}

{[}Description of theme with supporting quotes{]}

\begin{quote}
``Representative quote from participant'' (Participant ID)
\end{quote}

\subsubsection{Theme Integration}\label{theme-integration}

\begin{Shaded}
\begin{Highlighting}[]
\CommentTok{\# Thematic map visualization will be generated here}
\end{Highlighting}
\end{Shaded}

\textbf{Figure 4.5} illustrates the relationships between identified
themes.

\section{Summary of Results}\label{summary-of-results}

\subsection{Key Findings}\label{key-findings}

\begin{enumerate}
\def\labelenumi{\arabic{enumi}.}
\tightlist
\item
  \textbf{Research Question 1}: {[}Brief summary of findings and
  statistical significance{]}
\item
  \textbf{Research Question 2}: {[}Brief summary of findings and
  statistical significance{]}
\item
  \textbf{Research Question 3}: {[}Brief summary of findings and
  statistical significance{]}
\end{enumerate}

\subsection{Hypothesis Testing
Summary}\label{hypothesis-testing-summary}

\begin{table}

\caption{\label{tbl-hypothesis-summary}Summary of Hypothesis Testing
Results}

\centering{

\begin{Shaded}
\begin{Highlighting}[]
\CommentTok{\# Summary table of all hypotheses will be generated here}
\end{Highlighting}
\end{Shaded}

}

\end{table}%

\textbf{Table 4.8} summarizes the results of all hypothesis tests: -
\textbf{Hypothesis 1}: {[}Supported/Not supported{]} (p = {[}value{]}) -
\textbf{Hypothesis 2}: {[}Supported/Not supported{]} (p = {[}value{]}) -
\textbf{Hypothesis 3}: {[}Supported/Not supported{]} (p = {[}value{]})

\subsection{Effect Sizes}\label{effect-sizes}

All reported effect sizes follow Cohen's conventions: - \textbf{Small
effects}: {[}effect size measure{]} ≈ {[}threshold{]} - \textbf{Medium
effects}: {[}effect size measure{]} ≈ {[}threshold{]} - \textbf{Large
effects}: {[}effect size measure{]} ≈ {[}threshold{]}

The current study found {[}summary of effect sizes and their practical
significance{]}.

\section{Chapter Summary}\label{chapter-summary-2}

This chapter presented the results of the data analysis addressing the
three research questions. The findings indicate that {[}brief summary of
main findings{]}. {[}Summary of supported/unsupported hypotheses{]}.
These results provide the foundation for the discussion and
interpretation presented in the following chapter.

\bookmarksetup{startatroot}

\chapter{Discussion}\label{discussion}

This chapter interprets the findings presented in the Results chapter,
discusses their implications within the context of existing literature,
and addresses the study's limitations and contributions to the field.

\section{Summary of Key Findings}\label{summary-of-key-findings}

\subsection{Overview of Results}\label{overview-of-results}

The primary purpose of this study was to {[}restate main research
objective{]}. The investigation addressed three main research questions
through {[}brief methodological approach{]}. The key findings can be
summarized as follows:

\begin{enumerate}
\def\labelenumi{\arabic{enumi}.}
\tightlist
\item
  \textbf{Research Question 1}: {[}Brief summary of findings and whether
  hypothesis was supported{]}
\item
  \textbf{Research Question 2}: {[}Brief summary of findings and whether
  hypothesis was supported{]}
\item
  \textbf{Research Question 3}: {[}Brief summary of findings and whether
  hypothesis was supported{]}
\end{enumerate}

\subsection{Pattern of Results}\label{pattern-of-results}

The overall pattern of results {[}supports/partially supports/does not
support{]} the theoretical framework proposed in Chapter 2. {[}Brief
description of how results align or diverge from expectations{]}.

\section{Interpretation of Findings}\label{interpretation-of-findings}

\subsection{Research Question 1: {[}Question
Text{]}}\label{research-question-1-question-text-1}

\subsubsection{Findings in Context}\label{findings-in-context}

The finding that {[}summarize key result{]} is
{[}consistent/inconsistent{]} with previous research. Specifically, this
result {[}aligns with/contradicts{]} the work of {[}Author (Year){]} who
found {[}brief description of previous findings{]}.

\subsubsection{Theoretical Implications}\label{theoretical-implications}

These results provide {[}support/challenge{]} for {[}theoretical
framework/model{]}. The {[}significant/non-significant{]} relationship
between \hyperref[variables]{variables} suggests that {[}theoretical
interpretation{]}.

\textbf{Possible Mechanisms}: - \textbf{Mechanism 1}: {[}Description of
how this might work{]} - \textbf{Mechanism 2}: {[}Alternative
explanation{]} - \textbf{Mechanism 3}: {[}Additional perspective{]}

\subsubsection{Practical Implications}\label{practical-implications}

From a practical standpoint, these findings suggest that
\hyperref[practical-implications]{practical implications}: -
\textbf{Implication 1}: {[}Specific practical application{]} -
\textbf{Implication 2}: {[}Another practical consideration{]} -
\textbf{Implication 3}: {[}Additional practical insight{]}

\subsection{Research Question 2: {[}Question
Text{]}}\label{research-question-2-question-text-1}

\subsubsection{Findings in Context}\label{findings-in-context-1}

The {[}significant/non-significant{]} finding regarding {[}variable
relationship{]} extends previous research by {[}how it extends
knowledge{]}. This result is particularly noteworthy because {[}reason
for significance{]}.

\subsubsection{Comparison with Previous
Literature}\label{comparison-with-previous-literature}

\textbf{Consistent Findings}: - {[}Author (Year){]}: {[}How current
findings align{]} - {[}Author (Year){]}: {[}Additional consistent
finding{]}

\textbf{Divergent Findings}: - {[}Author (Year){]}: {[}How current
findings differ and possible reasons{]} - {[}Author (Year){]}:
{[}Additional divergent finding and explanation{]}

\subsubsection{Methodological
Considerations}\label{methodological-considerations-1}

The {[}methodology used{]} may have contributed to these findings in
several ways: - \textbf{Strength 1}: {[}How methodology enhanced
findings{]} - \textbf{Strength 2}: {[}Additional methodological
advantage{]} - \textbf{Limitation 1}: {[}How methodology may have
influenced results{]}

\subsection{Research Question 3: {[}Question
Text{]}}\label{research-question-3-question-text-1}

\subsubsection{Findings in Context}\label{findings-in-context-2}

The results for Research Question 3 revealed {[}summary of findings{]}.
This finding is {[}expected/surprising{]} given {[}context or previous
research{]}.

\subsubsection{Integration with Other
Findings}\label{integration-with-other-findings}

When considered alongside the results from RQ1 and RQ2, these findings
suggest {[}integrated interpretation{]}: - \textbf{Convergent evidence}:
{[}How findings support each other{]} - \textbf{Complementary insights}:
{[}How findings provide different perspectives{]} - \textbf{Potential
contradictions}: {[}Any conflicting results and possible explanations{]}

\section{Broader Implications}\label{broader-implications}

\subsection{Theoretical Contributions}\label{theoretical-contributions}

\subsubsection{Advancement of Theory}\label{advancement-of-theory}

This study contributes to {[}theoretical area{]} in several ways:

\begin{enumerate}
\def\labelenumi{\arabic{enumi}.}
\tightlist
\item
  \textbf{Theoretical Refinement}: {[}How findings refine existing
  theory{]}
\item
  \textbf{Conceptual Clarification}: {[}How findings clarify conceptual
  issues{]}
\item
  \textbf{Model Development}: {[}How findings inform model
  development{]}
\end{enumerate}

\subsubsection{New Theoretical Insights}\label{new-theoretical-insights}

The findings suggest several new theoretical considerations: -
\textbf{Insight 1}: {[}Novel theoretical perspective{]} -
\textbf{Insight 2}: {[}Additional theoretical contribution{]} -
\textbf{Insight 3}: {[}Further theoretical implication{]}

\subsection{Methodological
Contributions}\label{methodological-contributions}

\subsubsection{Methodological
Innovations}\label{methodological-innovations}

This study made several methodological contributions: -
\textbf{Innovation 1}: {[}Description of methodological advance{]} -
\textbf{Innovation 2}: {[}Additional methodological contribution{]} -
\textbf{Innovation 3}: {[}Further methodological insight{]}

\subsubsection{Validation of Approaches}\label{validation-of-approaches}

The study provides evidence for {[}methodological approach/instrument
validation{]}: - \textbf{Reliability}: {[}Evidence of reliability{]} -
\textbf{Validity}: {[}Evidence of validity{]} - \textbf{Utility}:
{[}Evidence of practical utility{]}

\subsection{Practical Applications}\label{practical-applications}

\subsubsection{Immediate Applications}\label{immediate-applications}

The findings have immediate relevance for: - \textbf{Practitioners}:
{[}How practitioners can use findings{]} - \textbf{Policymakers}:
{[}Policy implications{]} - \textbf{Educators}: {[}Educational
implications{]}

\subsubsection{Long-term Applications}\label{long-term-applications}

Over time, these findings may contribute to: - \textbf{System changes}:
{[}Broader systemic implications{]} - \textbf{Best practices}:
{[}Development of best practices{]} - \textbf{Future interventions}:
{[}Implications for intervention design{]}

\section{Limitations}\label{limitations-1}

\subsection{Methodological
Limitations}\label{methodological-limitations-1}

\subsubsection{Design Limitations}\label{design-limitations}

\begin{enumerate}
\def\labelenumi{\arabic{enumi}.}
\tightlist
\item
  \textbf{Cross-sectional design}: {[}Implications of design choice{]}

  \begin{itemize}
  \tightlist
  \item
    \textbf{Impact}: {[}How this affects interpretation{]}
  \item
    \textbf{Mitigation}: {[}How this was addressed{]}
  \end{itemize}
\item
  \textbf{Sampling limitations}: {[}Description of sampling
  constraints{]}

  \begin{itemize}
  \tightlist
  \item
    \textbf{Impact}: {[}Effect on generalizability{]}
  \item
    \textbf{Mitigation}: {[}Steps taken to address{]}
  \end{itemize}
\item
  \textbf{Measurement limitations}: {[}Instrument or measurement
  issues{]}

  \begin{itemize}
  \tightlist
  \item
    \textbf{Impact}: {[}Effect on validity{]}
  \item
    \textbf{Mitigation}: {[}Validation steps taken{]}
  \end{itemize}
\end{enumerate}

\subsubsection{Statistical Limitations}\label{statistical-limitations}

\begin{enumerate}
\def\labelenumi{\arabic{enumi}.}
\tightlist
\item
  \textbf{Power considerations}: {[}Issues with statistical power{]}
\item
  \textbf{Multiple comparisons}: {[}Potential Type I error inflation{]}
\item
  \textbf{Effect size interpretation}: {[}Limitations in effect size
  interpretation{]}
\end{enumerate}

\subsection{Conceptual Limitations}\label{conceptual-limitations}

\subsubsection{Theoretical Constraints}\label{theoretical-constraints}

\begin{enumerate}
\def\labelenumi{\arabic{enumi}.}
\tightlist
\item
  \textbf{Framework limitations}: {[}Constraints of theoretical
  framework{]}
\item
  \textbf{Variable selection}: {[}Limitations in variable selection{]}
\item
  \textbf{Causal inference}: {[}Limitations in causal claims{]}
\end{enumerate}

\subsubsection{Scope Limitations}\label{scope-limitations}

\begin{enumerate}
\def\labelenumi{\arabic{enumi}.}
\tightlist
\item
  \textbf{Population scope}: {[}Limitations in population
  generalizability{]}
\item
  \textbf{Contextual scope}: {[}Limitations in contextual
  generalizability{]}
\item
  \textbf{Temporal scope}: {[}Limitations in temporal
  generalizability{]}
\end{enumerate}

\subsection{External Validity
Considerations}\label{external-validity-considerations}

\subsubsection{Generalizability}\label{generalizability}

\textbf{Population Generalizability}: - \textbf{Strengths}: {[}Aspects
that enhance generalizability{]} - \textbf{Limitations}: {[}Factors that
limit generalizability{]}

\textbf{Ecological Validity}: - \textbf{Strengths}: {[}Real-world
applicability{]} - \textbf{Limitations}: {[}Artificial aspects of study
conditions{]}

\textbf{Temporal Validity}: - \textbf{Strengths}: {[}Contemporary
relevance{]} - \textbf{Limitations}: {[}Potential for temporal
changes{]}

\section{Future Research Directions}\label{future-research-directions}

\subsection{Immediate Research Needs}\label{immediate-research-needs}

\subsubsection{Replication Studies}\label{replication-studies}

\begin{enumerate}
\def\labelenumi{\arabic{enumi}.}
\tightlist
\item
  \textbf{Direct replication}: {[}Need for replication with similar
  methods{]}
\item
  \textbf{Conceptual replication}: {[}Need for replication with
  different methods{]}
\item
  \textbf{Extension studies}: {[}Need for studies extending current
  findings{]}
\end{enumerate}

\subsubsection{Methodological
Improvements}\label{methodological-improvements}

\begin{enumerate}
\def\labelenumi{\arabic{enumi}.}
\tightlist
\item
  \textbf{Longitudinal designs}: {[}Need for longitudinal approaches{]}
\item
  \textbf{Experimental designs}: {[}Need for experimental approaches{]}
\item
  \textbf{Mixed-methods approaches}: {[}Need for integrated
  methodologies{]}
\end{enumerate}

\subsection{Long-term Research Agenda}\label{long-term-research-agenda}

\subsubsection{Theoretical Development}\label{theoretical-development}

\begin{enumerate}
\def\labelenumi{\arabic{enumi}.}
\tightlist
\item
  \textbf{Theory building}: {[}Areas needing theoretical development{]}
\item
  \textbf{Model testing}: {[}Models requiring empirical testing{]}
\item
  \textbf{Integration efforts}: {[}Need for theoretical integration{]}
\end{enumerate}

\subsubsection{Practical Applications}\label{practical-applications-1}

\begin{enumerate}
\def\labelenumi{\arabic{enumi}.}
\tightlist
\item
  \textbf{Intervention development}: {[}Applications for intervention
  design{]}
\item
  \textbf{Program evaluation}: {[}Applications for program assessment{]}
\item
  \textbf{Policy research}: {[}Applications for policy development{]}
\end{enumerate}

\subsection{Specific Research
Questions}\label{specific-research-questions}

Based on the current findings, future research should address:

\begin{enumerate}
\def\labelenumi{\arabic{enumi}.}
\tightlist
\item
  \textbf{Question 1}: {[}Specific research question emerging from
  findings{]}
\item
  \textbf{Question 2}: {[}Additional research question{]}
\item
  \textbf{Question 3}: {[}Further research question{]}
\end{enumerate}

\section{Conclusion}\label{conclusion}

\subsection{Summary of Contributions}\label{summary-of-contributions}

This study makes several important contributions to
{[}field/discipline{]}:

\begin{enumerate}
\def\labelenumi{\arabic{enumi}.}
\tightlist
\item
  \textbf{Empirical contributions}: {[}Summary of empirical findings{]}
\item
  \textbf{Theoretical contributions}: {[}Summary of theoretical
  advances{]}
\item
  \textbf{Methodological contributions}: {[}Summary of methodological
  innovations{]}
\item
  \textbf{Practical contributions}: {[}Summary of practical
  implications{]}
\end{enumerate}

\subsection{Significance of Findings}\label{significance-of-findings}

The findings are significant because they: - \textbf{Advance
understanding}: {[}How findings advance knowledge{]} - \textbf{Challenge
assumptions}: {[}How findings challenge existing beliefs{]} -
\textbf{Inform practice}: {[}How findings inform professional
practice{]} - \textbf{Guide policy}: {[}How findings inform policy
decisions{]}

\subsection{Final Thoughts}\label{final-thoughts}

This research demonstrates that {[}key insight from the study{]}. While
limitations exist, the findings provide a foundation for {[}future
directions{]}. The study contributes to our understanding of {[}research
area{]} and offers practical insights for {[}relevant stakeholders{]}.

The complexity of {[}research phenomenon{]} requires continued
investigation, and this study provides a stepping stone toward
{[}ultimate research goal{]}. Future research building on these findings
will further advance our understanding and ability to {[}practical
outcome{]}.

\section{Chapter Summary}\label{chapter-summary-3}

This chapter interpreted the study's findings within the context of
existing literature and theory. The results {[}summary of how results
support or challenge existing knowledge{]}. Key contributions include
\hyperref[major-contributions]{major contributions}. While limitations
exist, particularly regarding {[}major limitations{]}, the findings
provide important insights for {[}theoretical/practical domains{]}.
Future research should focus on {[}key future directions{]} to build
upon these contributions and address remaining questions in the field.

\bookmarksetup{startatroot}

\chapter{Conclusion}\label{conclusion-1}

This final chapter provides a comprehensive summary of the dissertation,
synthesizes the key findings and contributions, and offers reflections
on the broader significance of the research. It concludes with final
thoughts on the implications for theory, practice, and future research.

\section{Dissertation Summary}\label{dissertation-summary}

\subsection{Research Problem and
Objectives}\label{research-problem-and-objectives}

This dissertation addressed the research problem of {[}statement of the
problem{]}. The study was motivated by {[}rationale for the research{]}
and aimed to {[}primary objective{]}.

The specific objectives were to: 1. {[}Objective 1{]} 2. {[}Objective
2{]} 3. {[}Objective 3{]}

\subsection{Theoretical Framework}\label{theoretical-framework-1}

The research was grounded in
\hyperref[theoretical-framework]{theoretical framework}, which provided
the conceptual foundation for understanding {[}phenomenon of
interest{]}. This framework suggested that {[}key theoretical
propositions{]} and guided the development of hypotheses and research
questions.

\subsection{Methodology Overview}\label{methodology-overview}

The study employed a \hyperref[research-design]{research design}
approach with {[}sample description{]}. Data were collected using
{[}data collection methods{]} and analyzed through {[}analytical
approaches{]}. The methodology was designed to ensure {[}key
methodological strengths{]} while acknowledging {[}key limitations{]}.

\subsection{Key Findings Synthesis}\label{key-findings-synthesis}

The research generated several important findings:

\textbf{Research Question 1}: {[}Brief summary of findings{]} - The
analysis revealed {[}key finding{]} - This finding
{[}supports/challenges{]} existing theory - Practical implications
include {[}implications{]}

\textbf{Research Question 2}: {[}Brief summary of findings{]} - The
results showed {[}key finding{]} - This contributes to understanding by
{[}contribution{]} - Applications include {[}applications{]}

\textbf{Research Question 3}: {[}Brief summary of findings{]} - The
investigation found {[}key finding{]} - This extends knowledge by
{[}extension{]} - Significance lies in {[}significance{]}

\section{Major Contributions}\label{major-contributions}

\subsection{Theoretical
Contributions}\label{theoretical-contributions-1}

\subsubsection{Advancement of Existing
Theory}\label{advancement-of-existing-theory}

This dissertation advances {[}theoretical area{]} in several significant
ways:

\begin{enumerate}
\def\labelenumi{\arabic{enumi}.}
\tightlist
\item
  \textbf{Theoretical Refinement}: The findings refine {[}specific
  theory{]} by {[}how it refines{]}
\item
  \textbf{Conceptual Clarification}: The research clarifies
  {[}conceptual issues{]} by demonstrating {[}clarification{]}
\item
  \textbf{Model Enhancement}: The results enhance {[}existing model{]}
  by {[}enhancement description{]}
\end{enumerate}

\subsubsection{Novel Theoretical
Insights}\label{novel-theoretical-insights}

The study contributes new theoretical understanding through:

\begin{enumerate}
\def\labelenumi{\arabic{enumi}.}
\tightlist
\item
  \textbf{New Relationships}: Discovery of {[}new relationship{]}
  between \hyperref[variables]{variables}
\item
  \textbf{Mechanism Identification}: Identification of {[}mechanism{]}
  that explains {[}phenomenon{]}
\item
  \textbf{Boundary Conditions}: Specification of {[}conditions{]} under
  which {[}theory applies{]}
\end{enumerate}

\subsection{Empirical Contributions}\label{empirical-contributions}

\subsubsection{Evidence Base}\label{evidence-base}

This research contributes to the empirical literature by:

\begin{enumerate}
\def\labelenumi{\arabic{enumi}.}
\tightlist
\item
  \textbf{Filling Gaps}: Addressing the gap in {[}specific area{]} by
  providing evidence that {[}evidence{]}
\item
  \textbf{Contradicting Assumptions}: Challenging the assumption that
  {[}assumption{]} by demonstrating {[}contradiction{]}
\item
  \textbf{Supporting Hypotheses}: Providing strong evidence for
  {[}hypothesis{]} across {[}contexts{]}
\end{enumerate}

\subsubsection{Methodological
Innovations}\label{methodological-innovations-1}

The study makes methodological contributions through:

\begin{enumerate}
\def\labelenumi{\arabic{enumi}.}
\tightlist
\item
  \textbf{Instrument Development}: Development/validation of
  {[}instrument{]} for measuring {[}construct{]}
\item
  \textbf{Analytical Approach}: Implementation of {[}analytical
  method{]} to address {[}methodological challenge{]}
\item
  \textbf{Design Innovation}: Use of {[}design approach{]} to overcome
  {[}limitation{]}
\end{enumerate}

\subsection{Practical Contributions}\label{practical-contributions}

\subsubsection{Immediate Applications}\label{immediate-applications-1}

The findings have direct practical relevance for:

\textbf{Practitioners}: - {[}Specific recommendation 1{]} - {[}Specific
recommendation 2{]} - {[}Specific recommendation 3{]}

\textbf{Organizations}: - {[}Organizational implication 1{]} -
{[}Organizational implication 2{]} - {[}Organizational implication 3{]}

\textbf{Policymakers}: - {[}Policy recommendation 1{]} - {[}Policy
recommendation 2{]} - {[}Policy recommendation 3{]}

\subsubsection{Long-term Impact}\label{long-term-impact}

Over time, this research may contribute to:

\begin{enumerate}
\def\labelenumi{\arabic{enumi}.}
\tightlist
\item
  \textbf{System-level Changes}: {[}Description of broader systemic
  implications{]}
\item
  \textbf{Best Practice Development}: {[}How findings inform best
  practices{]}
\item
  \textbf{Innovation}: {[}How findings might spur innovation{]}
\end{enumerate}

\section{Significance and Impact}\label{significance-and-impact}

\subsection{Academic Significance}\label{academic-significance}

\subsubsection{Disciplinary Impact}\label{disciplinary-impact}

This research makes important contributions to {[}discipline{]} by:

\begin{enumerate}
\def\labelenumi{\arabic{enumi}.}
\tightlist
\item
  \textbf{Advancing Knowledge}: {[}How knowledge is advanced{]}
\item
  \textbf{Methodological Progress}: {[}How methodology is advanced{]}
\item
  \textbf{Theoretical Development}: {[}How theory is developed{]}
\end{enumerate}

\subsubsection{Cross-disciplinary
Relevance}\label{cross-disciplinary-relevance}

The findings have implications beyond {[}primary discipline{]} for:

\begin{enumerate}
\def\labelenumi{\arabic{enumi}.}
\tightlist
\item
  \textbf{Related Field 1}: {[}Relevance to related field{]}
\item
  \textbf{Related Field 2}: {[}Additional cross-disciplinary
  relevance{]}
\item
  \textbf{Interdisciplinary Work}: {[}Implications for interdisciplinary
  research{]}
\end{enumerate}

\subsection{Societal Significance}\label{societal-significance}

\subsubsection{Social Impact}\label{social-impact}

The research addresses important social issues by:

\begin{enumerate}
\def\labelenumi{\arabic{enumi}.}
\tightlist
\item
  \textbf{Problem Addressing}: {[}How research addresses social
  problems{]}
\item
  \textbf{Equity Considerations}: {[}Implications for equity and
  justice{]}
\item
  \textbf{Public Benefit}: {[}How findings benefit society{]}
\end{enumerate}

\subsubsection{Economic Implications}\label{economic-implications}

The findings have economic significance through:

\begin{enumerate}
\def\labelenumi{\arabic{enumi}.}
\tightlist
\item
  \textbf{Cost Considerations}: {[}Economic implications of findings{]}
\item
  \textbf{Efficiency Gains}: {[}How findings improve efficiency{]}
\item
  \textbf{Resource Allocation}: {[}Implications for resource use{]}
\end{enumerate}

\subsection{Professional Relevance}\label{professional-relevance}

\subsubsection{Practice Enhancement}\label{practice-enhancement}

The research enhances professional practice by:

\begin{enumerate}
\def\labelenumi{\arabic{enumi}.}
\tightlist
\item
  \textbf{Evidence-Based Practice}: {[}How findings support
  evidence-based approaches{]}
\item
  \textbf{Professional Development}: {[}Implications for professional
  training{]}
\item
  \textbf{Quality Improvement}: {[}How findings improve service
  quality{]}
\end{enumerate}

\subsubsection{Policy Relevance}\label{policy-relevance}

The findings inform policy through:

\begin{enumerate}
\def\labelenumi{\arabic{enumi}.}
\tightlist
\item
  \textbf{Evidence for Policy}: {[}How findings provide policy
  evidence{]}
\item
  \textbf{Implementation Guidance}: {[}How findings guide
  implementation{]}
\item
  \textbf{Evaluation Frameworks}: {[}How findings inform evaluation{]}
\end{enumerate}

\section{Limitations and
Acknowledgments}\label{limitations-and-acknowledgments}

\subsection{Research Limitations}\label{research-limitations}

While this study makes important contributions, several limitations
should be acknowledged:

\subsubsection{Methodological
Limitations}\label{methodological-limitations-2}

\begin{enumerate}
\def\labelenumi{\arabic{enumi}.}
\tightlist
\item
  \textbf{Design Constraints}: {[}Acknowledgment of design
  limitations{]}
\item
  \textbf{Sampling Limitations}: {[}Recognition of sampling
  constraints{]}
\item
  \textbf{Measurement Limitations}: {[}Acknowledgment of measurement
  issues{]}
\end{enumerate}

\subsubsection{Scope Limitations}\label{scope-limitations-1}

\begin{enumerate}
\def\labelenumi{\arabic{enumi}.}
\tightlist
\item
  \textbf{Generalizability}: {[}Limitations in generalizability{]}
\item
  \textbf{Context Specificity}: {[}Contextual limitations{]}
\item
  \textbf{Temporal Constraints}: {[}Time-related limitations{]}
\end{enumerate}

\subsection{Strengths and Mitigations}\label{strengths-and-mitigations}

Despite limitations, the study has several strengths:

\begin{enumerate}
\def\labelenumi{\arabic{enumi}.}
\tightlist
\item
  \textbf{Methodological Rigor}: {[}Description of methodological
  strengths{]}
\item
  \textbf{Comprehensive Approach}: {[}Description of comprehensive
  elements{]}
\item
  \textbf{Practical Relevance}: {[}Description of practical value{]}
\end{enumerate}

\section{Future Research Agenda}\label{future-research-agenda}

\subsection{Immediate Research
Priorities}\label{immediate-research-priorities}

\subsubsection{Replication and
Extension}\label{replication-and-extension}

\begin{enumerate}
\def\labelenumi{\arabic{enumi}.}
\tightlist
\item
  \textbf{Replication Studies}: {[}Need for replication in different
  contexts{]}
\item
  \textbf{Extension Research}: {[}Areas for extending current
  findings{]}
\item
  \textbf{Validation Studies}: {[}Need for validation of findings{]}
\end{enumerate}

\subsubsection{Methodological Advances}\label{methodological-advances}

\begin{enumerate}
\def\labelenumi{\arabic{enumi}.}
\tightlist
\item
  \textbf{Longitudinal Research}: {[}Need for longitudinal approaches{]}
\item
  \textbf{Experimental Studies}: {[}Need for experimental designs{]}
\item
  \textbf{Mixed-Methods Research}: {[}Need for integrated approaches{]}
\end{enumerate}

\subsection{Long-term Research Vision}\label{long-term-research-vision}

\subsubsection{Theoretical Development}\label{theoretical-development-1}

\begin{enumerate}
\def\labelenumi{\arabic{enumi}.}
\tightlist
\item
  \textbf{Theory Building}: {[}Areas needing theoretical development{]}
\item
  \textbf{Model Testing}: {[}Models requiring empirical testing{]}
\item
  \textbf{Integration Efforts}: {[}Need for theoretical integration{]}
\end{enumerate}

\subsubsection{Applied Research}\label{applied-research}

\begin{enumerate}
\def\labelenumi{\arabic{enumi}.}
\tightlist
\item
  \textbf{Intervention Studies}: {[}Need for intervention research{]}
\item
  \textbf{Implementation Research}: {[}Need for implementation
  studies{]}
\item
  \textbf{Evaluation Research}: {[}Need for evaluation studies{]}
\end{enumerate}

\subsection{Specific Research
Questions}\label{specific-research-questions-1}

Future research should address:

\begin{enumerate}
\def\labelenumi{\arabic{enumi}.}
\tightlist
\item
  \textbf{Question 1}: {[}Specific future research question{]}
\item
  \textbf{Question 2}: {[}Additional research question{]}
\item
  \textbf{Question 3}: {[}Further research question{]}
\end{enumerate}

\section{Personal Reflections}\label{personal-reflections}

\subsection{Research Journey}\label{research-journey}

This doctoral journey has been {[}personal reflection on the research
process{]}. The process of conducting this research has {[}insights
gained through the process{]}.

\subsection{Lessons Learned}\label{lessons-learned}

Key lessons from this research experience include:

\begin{enumerate}
\def\labelenumi{\arabic{enumi}.}
\tightlist
\item
  \textbf{Methodological Lessons}: {[}What was learned about
  methodology{]}
\item
  \textbf{Theoretical Lessons}: {[}What was learned about theory{]}
\item
  \textbf{Practical Lessons}: {[}What was learned about practice{]}
\end{enumerate}

\subsection{Growth and Development}\label{growth-and-development}

This research has contributed to personal and professional growth
through:

\begin{enumerate}
\def\labelenumi{\arabic{enumi}.}
\tightlist
\item
  \textbf{Skill Development}: {[}Skills developed through research{]}
\item
  \textbf{Knowledge Expansion}: {[}How knowledge was expanded{]}
\item
  \textbf{Professional Identity}: {[}How professional identity
  evolved{]}
\end{enumerate}

\section{Final Thoughts}\label{final-thoughts-1}

\subsection{Research Significance}\label{research-significance}

This dissertation represents a significant contribution to {[}field{]}
through \hyperref[major-contributions]{major contributions}. The
research addresses important questions about {[}research focus{]} and
provides evidence that {[}key evidence{]}.

\subsection{Broader Impact}\label{broader-impact}

The implications of this work extend beyond academic circles to
\hyperref[broader-implications]{broader implications}. The findings
offer practical guidance for {[}stakeholders{]} and contribute to
{[}broader goals{]}.

\subsection{Call to Action}\label{call-to-action}

This research calls for {[}specific actions{]}:

\begin{enumerate}
\def\labelenumi{\arabic{enumi}.}
\tightlist
\item
  \textbf{For Researchers}: {[}What researchers should do{]}
\item
  \textbf{For Practitioners}: {[}What practitioners should do{]}
\item
  \textbf{For Policymakers}: {[}What policymakers should do{]}
\end{enumerate}

\subsection{Concluding Statement}\label{concluding-statement}

In conclusion, this dissertation has examined {[}research focus{]}
through {[}approach{]}. The findings demonstrate {[}key demonstration{]}
and contribute to {[}contribution{]}. While challenges remain, this
research provides a foundation for {[}future work{]} and offers hope for
{[}ultimate goal{]}.

The complexity of {[}phenomenon{]} requires continued investigation,
collaboration, and innovation. This study represents one step in the
ongoing effort to understand and address {[}research problem{]}. It is
my hope that this work will inspire others to continue this important
research and contribute to {[}ultimate vision{]}.

\section{Chapter Summary}\label{chapter-summary-4}

This concluding chapter synthesized the major findings and contributions
of the dissertation. The research addressed {[}research problem{]}
through \hyperref[methodology]{methodology} and found
\hyperref[key-findings]{key findings}. Major contributions include
\hyperref[theoretical-contributions]{theoretical contributions},
\hyperref[empirical-contributions]{empirical contributions}, and
\hyperref[practical-contributions]{practical contributions}. While
limitations exist, the study provides important insights for
{[}stakeholders{]} and establishes directions for future research. The
work represents a significant contribution to {[}field{]} and offers
practical guidance for addressing {[}practical problems{]}. Future
research should build upon these findings to further advance
understanding and practice in {[}area{]}.

\bookmarksetup{startatroot}

\chapter*{References}\label{references}
\addcontentsline{toc}{chapter}{References}

\markboth{References}{References}

\phantomsection\label{refs}
\begin{CSLReferences}{1}{0}
\bibitem[\citeproctext]{ref-example2024}
Author, F. (2024). \emph{Example reference}. Example Publisher.

\bibitem[\citeproctext]{ref-sample2023}
Sample, A. (2023). Sample article title. \emph{Journal Name},
\emph{1}(1), 1--10.

\end{CSLReferences}

\cleardoublepage
\phantomsection
\addcontentsline{toc}{part}{Appendices}
\appendix

\chapter*{Appendix A: Additional
Materials}\label{appendix-a-additional-materials}
\addcontentsline{toc}{chapter}{Appendix A: Additional Materials}

\markboth{Appendix A: Additional Materials}{Appendix A: Additional
Materials}

\section*{Survey Instruments}\label{survey-instruments}
\addcontentsline{toc}{section}{Survey Instruments}

\markright{Survey Instruments}

\subsection*{Primary Survey Instrument}\label{primary-survey-instrument}
\addcontentsline{toc}{subsection}{Primary Survey Instrument}

{[}Include full text of survey instruments used in the study{]}

\textbf{Instructions}: {[}Survey instructions given to participants{]}

\textbf{Items}: 1. {[}Survey item 1{]} 2. {[}Survey item 2{]} 3.
{[}Survey item 3{]}

\subsection*{Secondary Instruments}\label{secondary-instruments-1}
\addcontentsline{toc}{subsection}{Secondary Instruments}

{[}Include any additional instruments, scales, or measures{]}

\section*{Additional Statistical
Output}\label{additional-statistical-output}
\addcontentsline{toc}{section}{Additional Statistical Output}

\markright{Additional Statistical Output}

\subsection*{Supplementary Analyses}\label{supplementary-analyses}
\addcontentsline{toc}{subsection}{Supplementary Analyses}

{[}Include additional statistical output that supports but doesn't fit
in main results{]}

\subsection*{Model Diagnostics}\label{model-diagnostics}
\addcontentsline{toc}{subsection}{Model Diagnostics}

{[}Include diagnostic plots and tests not shown in main chapters{]}

\section*{Interview Protocols (if
applicable)}\label{interview-protocols-if-applicable}
\addcontentsline{toc}{section}{Interview Protocols (if applicable)}

\markright{Interview Protocols (if applicable)}

\subsection*{Semi-structured Interview
Guide}\label{semi-structured-interview-guide}
\addcontentsline{toc}{subsection}{Semi-structured Interview Guide}

\textbf{Opening Questions}: - {[}Opening question 1{]} - {[}Opening
question 2{]}

\textbf{Main Questions}: - {[}Main question 1{]} - {[}Main question 2{]}

\textbf{Closing Questions}: - {[}Closing question 1{]} - {[}Closing
question 2{]}

\section*{Supplementary Data}\label{supplementary-data}
\addcontentsline{toc}{section}{Supplementary Data}

\markright{Supplementary Data}

\subsection*{Additional Tables}\label{additional-tables}
\addcontentsline{toc}{subsection}{Additional Tables}

{[}Include supplementary tables referenced in main text{]}

\subsection*{Additional Figures}\label{additional-figures}
\addcontentsline{toc}{subsection}{Additional Figures}

{[}Include supplementary figures referenced in main text{]}




\end{document}
